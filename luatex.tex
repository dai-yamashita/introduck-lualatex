% +++
% engine="lualatex"
% +++

\documentclass[main]{subfiles}

\begin{document}
\begin{abstract}
本書は\LuaLaTeX{}の入門書を目指した書籍となります。
では、\LuaLaTeX{}とは何なのでしょうか。

\end{abstract}

\LuaLaTeX{}では\pLaTeX{}では使えない・使い辛いパッケージを扱うことができます。
それは高機能な図形描画パッケージの機能の全てであったり、
OpenTypeフォントの細かい機能であったりします。
これは\LuaLaTeX{}がPDFを直接出力できるという特性と、
後発の\TeX{}派生エンジンであることが大きく関与します。

そもそも何故\pLaTeX{}がPDFを直接出してくれないのかというと、
PDFが普及する以前に生まれた\LaTeX{}の派生だという要因があります。

\TeX{}とその派生などについては\appendref{chap:history}で触れますので、
「\LuaLaTeX{}は新しいから色々便利になってるんだなあ」とを思ってもらえれば
バッチリです。本章ではPDF出力についての補足と、\LuaTeX{}-jaチームについて紹介します。

\section{\TeX{}はPDFよりも古い}

\pLaTeX{}を使ってPDFを作るときにどのようなコマンドを叩いているでしょうか。
多くの場合、PDFを得るのを目的として
\termref{term:platex}のようにしているかと思います。
\cmd{dvipdfmx <ドキュメント>.tex}のような失敗をしたことがある方もいるかもしれません。
p\LaTeX{}では先ず\ext{dvi}ファイルを生成します。dvipdfmxという別の
ツールが実際のPDFを作成します。DVIという形式は以下のような特徴があります。
\begin{itemize}
\item{フォントは埋め込まれていない}
\item{様々な出力のための中間形式}
\end{itemize}

Re:VIEWやPandocの層で\LaTeX{}をバックエンドにしていると
「PDFになる前に謎の中間形式がある」くらいの感覚の方もいるかもしれませんが、
そもそもPDFにするだけが\LaTeX{}の出力ではありません。


\terminput{./terminal/platex.tex}[caption="p\LaTeXでPDFを作るコマンド", label="term:platex"]

\begin{column}[Luajit\LaTeX{}]
Luajit\TeX{} は \LuaTeX{}で使用されている Lua を LuaJIT に置き換えたものです。
JIT(Just In Time)コンパイラはインタプリタ型の言語を適時機械語にしていくことで
高速に動作
させることを目的としたものですが、Luaスクリプト部分を機械語に置き換える
処理自体の重さなどを要することになるので、
ユースケースによっては体感的なパフォーマンスを得られなかったりします。
また、
luaotfloadのサポートにはLuajit\TeX{}は含まれていないようです\cite{luaotfload:issue:jit}
ので、2019年9月現在、\LuaLaTeX{}でビルドできた文書がLuajit\LaTeX{}
では動作しない事例などが報告されています\cite{texforum:jit}。
\end{column}

\section{\LuaTeX{}-ja}
\LuaTeX{}-jaは\LuaTeX{}-jaチームによって開発されている、

\pTeX{}はエンジン(\TeX{}処理系)による日本語環境の実装ですが、
\LuaTeX{}-jaはマクロによる実装であること
\footnote{\appendref{chap:luatex}でこの違いについて補足します。}、
\pTeX{}での不自然な仕様・挙動に関しては改めていく方針で
あるため\cite{luatexja:doc}、
\pLaTeX{}の文書ファイルをそのまま持ってきても動作しない箇所があります。

単位系としては\code{zh}\index{zh}・\code{zw}\index{zw}は
\code{\zh}\index{zh@\code{\zh}}・\code{\zw}\index{zw@\code{\zw}}で、
\code{H}\index{H}・\code{Q}\index{Q}は
\code{\jH}\index{jH@\code{\jH}}、\code{\jQ}\index{jQ@\code{\jQ}}の制御綴で表すなどです。

和文文字改行直後の空白の扱いについてなど、エンジン拡張とマクロ拡張の違いによる挙動の違い
の詳細は\LuaTeX{}-jaのドキュメント\cite{luatexja:doc}を参照してください。

この他\LuaTeX{}での、listings\index{listings}を使ったときの和文の制御や
縦書きクラスでのgeometry\index{geometry}へのパッチ処理など、見えにくいところでも活躍しています。

\section{バージョン}
本書が想定する\TeX{}Liveバージョンである
\TeX{}Live2019が提供する\LuaTeX{}のバージョンは1.10.0であり、
内蔵のLua処理系は5.3.5となります\footnote{LuaJITのLuaは5.1なので、%
Luajit\TeX{}ではこれに由来するコケ方をする場合があります。}。
それ以前から収録されていたものの、
2016年に1.0.0がリリースされ、ある程度安定して動作するようになり
\TeX{}のコアなユーザ、開発者でない層にも知られるようになってきました。
とはいえアグレッシブな変更も未だそこそこ多く、
インターネット上の情報などを参考にする際は情報が発信された日付に
注意する必要があります\footnote{\LuaLaTeX{}に限った話ではありませんが。}。
それでは商用に耐えないのかというとそんなことはありません。
「数式組版」\cite{kieda:math}などは\LuaLaTeX{}で組まれているそうです\cite{kieda:tw}。

ついでに本書も\LuaLaTeX{}直書きで組んでいます。
草稿は軽量マークアップで行えるように準備するべきだと痛感しました。

\begin{column}[\ConTeXt{}]
\LaTeX{}は\TeX{}エンジンの上で動作するフレームワークのようなもの
で、他にも\TeX{}エンジン上で動作させられるフレームワークが存在します。
それが\ConTeXt{}です。クラスファイルを読み込むコマンド1つ取っても、
書き方が\LaTeX{}とはかなり違います。

\LuaTeX{}の更新による動作の変更がアグレッシブで\LaTeX{}ユーザを悩ませる原因の1つに、
\LuaTeX{}の開発チームが力を入れているのが\ConTeXt{}の方であるということも挙げられます。
\end{column}

\section{LuaHB\LaTeX{}}
次期バージョン\LuaLaTeX{}は\LuaTeX{}ではなくLuaHB\TeX{}がエンジンになる
という説が濃厚です。\TeX{}Liveに設定を追加すると使用可能になる、次期バージョンの
lualatex-devを調べると\code{This is LuaHBTeX, Version 1.11.2 (TeX Live 2020/dev)}と
表示されます。
HBはHarfBuzz\cite{harfbuzz}\index{HarfBuzz}の略です。LuaHB\TeX{}は\LuaTeX{}に
HarfBuzzを組み込んだエンジンということです。これでどう便利になるのかについては
\chapref{chap:fontsetting}で触れます。







\end{document}